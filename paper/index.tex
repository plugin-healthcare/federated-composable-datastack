% Options for packages loaded elsewhere
\PassOptionsToPackage{unicode}{hyperref}
\PassOptionsToPackage{hyphens}{url}
\PassOptionsToPackage{dvipsnames,svgnames,x11names}{xcolor}
%
\documentclass[
  letterpaper,
  DIV=11,
  numbers=noendperiod]{scrartcl}

\usepackage{amsmath,amssymb}
\usepackage{iftex}
\ifPDFTeX
  \usepackage[T1]{fontenc}
  \usepackage[utf8]{inputenc}
  \usepackage{textcomp} % provide euro and other symbols
\else % if luatex or xetex
  \usepackage{unicode-math}
  \defaultfontfeatures{Scale=MatchLowercase}
  \defaultfontfeatures[\rmfamily]{Ligatures=TeX,Scale=1}
\fi
\usepackage{lmodern}
\ifPDFTeX\else  
    % xetex/luatex font selection
\fi
% Use upquote if available, for straight quotes in verbatim environments
\IfFileExists{upquote.sty}{\usepackage{upquote}}{}
\IfFileExists{microtype.sty}{% use microtype if available
  \usepackage[]{microtype}
  \UseMicrotypeSet[protrusion]{basicmath} % disable protrusion for tt fonts
}{}
\makeatletter
\@ifundefined{KOMAClassName}{% if non-KOMA class
  \IfFileExists{parskip.sty}{%
    \usepackage{parskip}
  }{% else
    \setlength{\parindent}{0pt}
    \setlength{\parskip}{6pt plus 2pt minus 1pt}}
}{% if KOMA class
  \KOMAoptions{parskip=half}}
\makeatother
\usepackage{xcolor}
\setlength{\emergencystretch}{3em} % prevent overfull lines
\setcounter{secnumdepth}{-\maxdimen} % remove section numbering
% Make \paragraph and \subparagraph free-standing
\makeatletter
\ifx\paragraph\undefined\else
  \let\oldparagraph\paragraph
  \renewcommand{\paragraph}{
    \@ifstar
      \xxxParagraphStar
      \xxxParagraphNoStar
  }
  \newcommand{\xxxParagraphStar}[1]{\oldparagraph*{#1}\mbox{}}
  \newcommand{\xxxParagraphNoStar}[1]{\oldparagraph{#1}\mbox{}}
\fi
\ifx\subparagraph\undefined\else
  \let\oldsubparagraph\subparagraph
  \renewcommand{\subparagraph}{
    \@ifstar
      \xxxSubParagraphStar
      \xxxSubParagraphNoStar
  }
  \newcommand{\xxxSubParagraphStar}[1]{\oldsubparagraph*{#1}\mbox{}}
  \newcommand{\xxxSubParagraphNoStar}[1]{\oldsubparagraph{#1}\mbox{}}
\fi
\makeatother


\providecommand{\tightlist}{%
  \setlength{\itemsep}{0pt}\setlength{\parskip}{0pt}}\usepackage{longtable,booktabs,array}
\usepackage{calc} % for calculating minipage widths
% Correct order of tables after \paragraph or \subparagraph
\usepackage{etoolbox}
\makeatletter
\patchcmd\longtable{\par}{\if@noskipsec\mbox{}\fi\par}{}{}
\makeatother
% Allow footnotes in longtable head/foot
\IfFileExists{footnotehyper.sty}{\usepackage{footnotehyper}}{\usepackage{footnote}}
\makesavenoteenv{longtable}
\usepackage{graphicx}
\makeatletter
\newsavebox\pandoc@box
\newcommand*\pandocbounded[1]{% scales image to fit in text height/width
  \sbox\pandoc@box{#1}%
  \Gscale@div\@tempa{\textheight}{\dimexpr\ht\pandoc@box+\dp\pandoc@box\relax}%
  \Gscale@div\@tempb{\linewidth}{\wd\pandoc@box}%
  \ifdim\@tempb\p@<\@tempa\p@\let\@tempa\@tempb\fi% select the smaller of both
  \ifdim\@tempa\p@<\p@\scalebox{\@tempa}{\usebox\pandoc@box}%
  \else\usebox{\pandoc@box}%
  \fi%
}
% Set default figure placement to htbp
\def\fps@figure{htbp}
\makeatother

\KOMAoption{captions}{tableheading}
\makeatletter
\@ifpackageloaded{caption}{}{\usepackage{caption}}
\AtBeginDocument{%
\ifdefined\contentsname
  \renewcommand*\contentsname{Table of contents}
\else
  \newcommand\contentsname{Table of contents}
\fi
\ifdefined\listfigurename
  \renewcommand*\listfigurename{List of Figures}
\else
  \newcommand\listfigurename{List of Figures}
\fi
\ifdefined\listtablename
  \renewcommand*\listtablename{List of Tables}
\else
  \newcommand\listtablename{List of Tables}
\fi
\ifdefined\figurename
  \renewcommand*\figurename{Figure}
\else
  \newcommand\figurename{Figure}
\fi
\ifdefined\tablename
  \renewcommand*\tablename{Table}
\else
  \newcommand\tablename{Table}
\fi
}
\@ifpackageloaded{float}{}{\usepackage{float}}
\floatstyle{ruled}
\@ifundefined{c@chapter}{\newfloat{codelisting}{h}{lop}}{\newfloat{codelisting}{h}{lop}[chapter]}
\floatname{codelisting}{Listing}
\newcommand*\listoflistings{\listof{codelisting}{List of Listings}}
\makeatother
\makeatletter
\makeatother
\makeatletter
\@ifpackageloaded{caption}{}{\usepackage{caption}}
\@ifpackageloaded{subcaption}{}{\usepackage{subcaption}}
\makeatother

\usepackage{bookmark}

\IfFileExists{xurl.sty}{\usepackage{xurl}}{} % add URL line breaks if available
\urlstyle{same} % disable monospaced font for URLs
\hypersetup{
  pdftitle={An open-source reference implementation of a federated composable datastack},
  pdfauthor={Yannick Vinkesteijn; Daniel Kapitan},
  pdfkeywords={data engineering, lakehouse, Apache Arrow, Apache
Parquet, Apache Icebert, duckdb, polars},
  colorlinks=true,
  linkcolor={blue},
  filecolor={Maroon},
  citecolor={Blue},
  urlcolor={Blue},
  pdfcreator={LaTeX via pandoc}}


\title{An open-source reference implementation of a federated composable
datastack}
\author{Yannick Vinkesteijn \and Daniel Kapitan}
\date{}

\begin{document}
\maketitle
\begin{abstract}
We describe \ldots{}
\end{abstract}


(intented for publication in something like a \emph{practice} paper in
\href{https://cacm.acm.org/author-guidelines/\#CACMsubmission}{Communications
of ACM})

\subsection{A confluence of developments towards federated analytical
data
systems}\label{a-confluence-of-developments-towards-federated-analytical-data-systems}

We observe a confluence of various trends into what we call federated
analytics data systems(FADS). In Europe, ongoing effort to implement
data spaces in various industries, including mobility, healthcare and
energy. Given the need for secure and trustworthy data sharing, much
work is underway to design and implement FADS On the user side, we see
that trends such as data spaces. At the same time, the computer science
and engineering has development couple of new concepts and technologies
that make it easier to implement FADS:

\begin{itemize}
\tightlist
\item
  A data mesh has been developed as an alternative to centralized data
  warehouse systems. Data mesh conceptualizes data as a product which
  can be consumed through an application programming interface (API).
  Although originally coined in the context of enterprise data
  platforms, it can be readily extend beyond the boundaries of a single
  organization into a data space
\item
  Capabilities of edge computing and single-node computing has increased
  significantly, whereby it is now possible to process up to 1 TB of
  tabular data on a single node
\item
  The composable data stack provides a way to unbundle the venerable
  data base into loosely components, thereby making it easier and more
  practical to implement FADS
\item
  The increasing need for privacy-enhancing technologies (PETs)
  additionally fuels the development of FADS-related technologies, where
  technologies such as secure multi-party computation (MPC) are now
  sufficiently mature to be used on an industrial scale.
\item
  Federated machine learming (or federated learning in short) has
  matured as a means for decentralized training of predictive models,
  most notable through weights sharing of deep learning networks
\end{itemize}

As noted by Perreira et
al.~(\href{https://dl.acm.org/doi/10.14778/3603581.3603604}{2023}),
however:

\begin{quote}
The requirement for specialization in data management systems has
evolved faster than our software development practices. After decades of
organic growth, this situation has created a siloed landscape composed
of hundreds of products developed and maintained as monoliths, with
limited reuse between systems. This fragmentation has resulted in
developers often reinventing the wheel, increased maintenance costs, and
slowed down innovation. It has also affected the end users, who are
often required to learn the idiosyncrasies of dozens of incompatible SQL
and non-SQL API dialects, and settle for systems with incomplete
functionality and inconsistent semantics.
\end{quote}

This paper rises to their call to take a principled, open source
approach to FADS aimed at ``\ldots standardizing different aspects of
the data stack (\ldots) advocating for a paradigm shift in how data
management systems are designed'', focusing on federated analytics data
systems.

\begin{enumerate}
\def\labelenumi{\arabic{enumi}.}
\tightlist
\item
  functional architecture and specification of FADS that integrates
  various common practices and blueprints from the data engineering
  community;
\item
  a reference implementation in the Python-Rust data stack that is
  quickly emerging as a new \emph{de facto} standard for performant and
  reliable analytical processing;
\item
  implementation of functionality specific to healthcare, which provided
  the impetus for this work (kapitan2025data).
\end{enumerate}

\subsection{Desiderata of analytical data
systems}\label{desiderata-of-analytical-data-systems}

We take Klepmann (2017) as our starting point, who states that ``Many
applications today are~\emph{data-intensive}, as opposed
to~\emph{compute-intensive}.~Raw CPU power is rarely a limiting factor
for these applications---bigger problems are usually the amount of data,
the complexity of data, and the speed at which it is changing.''

Generically, we want:

\begin{longtable}[]{@{}
  >{\raggedright\arraybackslash}p{(\linewidth - 4\tabcolsep) * \real{0.3491}}
  >{\raggedright\arraybackslash}p{(\linewidth - 4\tabcolsep) * \real{0.2925}}
  >{\raggedright\arraybackslash}p{(\linewidth - 4\tabcolsep) * \real{0.3585}}@{}}
\toprule\noalign{}
\begin{minipage}[b]{\linewidth}\raggedright
Reliability
\end{minipage} & \begin{minipage}[b]{\linewidth}\raggedright
Scalability
\end{minipage} & \begin{minipage}[b]{\linewidth}\raggedright
Maintainability
\end{minipage} \\
\midrule\noalign{}
\endhead
\bottomrule\noalign{}
\endlastfoot
tolerating hardware \& software vaults & Measuring load \& performance &
Operability, simplicity \& evolvability \\
human error & Latency percentiles, throughput & \\
\end{longtable}

We focus on analytical data systems, with different patterns from
transactional data systems.

\begin{longtable}[]{@{}
  >{\raggedright\arraybackslash}p{(\linewidth - 4\tabcolsep) * \real{0.3333}}
  >{\raggedright\arraybackslash}p{(\linewidth - 4\tabcolsep) * \real{0.3333}}
  >{\raggedright\arraybackslash}p{(\linewidth - 4\tabcolsep) * \real{0.3333}}@{}}
\toprule\noalign{}
\begin{minipage}[b]{\linewidth}\raggedright
Property
\end{minipage} & \begin{minipage}[b]{\linewidth}\raggedright
Transaction processing systems (OLTP)
\end{minipage} & \begin{minipage}[b]{\linewidth}\raggedright
Analytic systems (OLAP)
\end{minipage} \\
\midrule\noalign{}
\endhead
\bottomrule\noalign{}
\endlastfoot
Main read pattern & Small number of records per query, fetched by key &
Aggregate over large number of records \\
Main write pattern & Random-access, low-latency writes from user input &
Bulk import (ETL) or event stream \\
Primarily used by & End user/customer, via web application & Internal
analyst, for decision support \\
What data represents & Latest state of data (current point in time) &
History of events that happened over time \\
Dataset size & Gigabytes to terabytes & Terabytes to petabytes \\
\end{longtable}

\subsection{Design principles of analytical data
systems}\label{design-principles-of-analytical-data-systems}

These functional requirements lead us to the following design principles

\begin{itemize}
\tightlist
\item
  \textbf{Colum-oriented storage and memory layout:} Apache Arrow
  ecosystem, including Apache Flight
\item
  \textbf{Late-binding with logical data models most suited for
  analytics:} ELT pattern with zonal architecture

  \begin{itemize}
  \tightlist
  \item
    \emph{staging zone:} hard business rules (does incoming data comply
    to syntactic standard), change data capture
  \item
    \emph{linkage \& conformity zone:} concept-oriented tables,
    typically following a data vault modeling principle, ascertain
    referential integrity across resources, with tables per concept and
    linking tables. Mapping to coding systems.
  \item
    \emph{consumption zone:} convenient standardized views like an event
    table (patient journey, layout for process mining) and
    dimenreferention integrity across star schema
  \end{itemize}
\end{itemize}

\subsection{Reference implementation with current open source software
(OSS)
components}\label{reference-implementation-with-current-open-source-software-oss-components}

At the lower, technical, levels we follow the rationale of the
composable data stack - Python and SQL(-like) languages as the \emph{de
facto} standard for analytical processing i.e.~the most commonly used
analytical scripting languages. Where necessary, using Intermediate
Representations (IR), any analytical query can be transpiled to the
target engine of choice - Single-node compute capable of efficiently
processing up to 1 TB of data within tens of seconds (polars, DuckDB),
so we do away with distributed processing - Open table formats (Iceberg,
Hudi, Delta) and open file formats (parquet, AVRO)

\subsection{Bringing it all together for federated analytics \& machine
learning
(FAML)}\label{bringing-it-all-together-for-federated-analytics-machine-learning-faml}

\begin{itemize}
\tightlist
\item
  Local data stations are conceptualized as serverless lakehouses

  \begin{itemize}
  \tightlist
  \item
    Local ELT pipelines
  \item
    Decentralized (pre-)processing, including quality control upon
    ingest
  \item
    \ldots{}
  \end{itemize}
\item
  For horizontally partioned data, we can apply FAML techniques where
  only aggregated results are combined centrally
\item
  For vertically partitioned data, we need an intermediate/temporary
  zone for linking the data
\item
  For both horizontally and vertically partitioned data, we can choose
  to add PETs, most specifically MPC, as an extra security measure

  \begin{itemize}
  \tightlist
  \item
    Horizontally partitioned data:
    \href{https://rosemanlabs.com/en/blogs/privacy-safe-federated-learning}{one-short
    FL}
  \item
    Vertically partitioned data: linkage in the blind
  \end{itemize}
\item
  standardized approach to mapppings
\end{itemize}




\end{document}
